%%
%%  Example paper
%%
%%

%%%%%%%%%%%%%%%%%% Usenix style %%%%%%%%%%%%%%%%%%%%%%%%%%%%%%%%%
\documentclass[10pt,twocolumn,a4paper]{article}
\usepackage{styles/usenix-style}

\author{Jean Diestl}

%%%%%%%%%%%%%%%%%% Document %%%%%%%%%%%%%%%%%%%%%%%%%%%%%%%%%%%%%%%%%%%
% TODO: Change draft to final before submitting final version.
\usepackage[draft]{styles/ka-style}
\usepackage{cite,xspace,ifthen,graphicx,listings,lstautogobble}

\usepackage[
   pdfauthor={Jean Diestl},
   pdftitle={Linux namespaces},
   pdfsubject={KIT testing OS Level Virtualization},
   pdfkeywords={Papers, Template}
]{hyperref}

\begin{document}

\title{Linux Namespaces: KIT testing OS Level Virtualization}

\newcommand{\todo}[1]{{\texttt{[#1]}}}
\newcommand{\code}[1]{{\tt \small{#1}}}

\maketitle
%\draftfooter
\lstset{language=C, basicstyle=\ttfamily,
        columns=fullflexible,
        string=[b]', showspaces=false, showtabs=false,
         captionpos=b,
        framerule=1pt,
        breaklines=true, breakatwhitespace=true,
        autogobble=true}


\begin{abstract}
All operating systems virtualize hardware like memory or the cpu but other resources remain shared between processes.
In kernel 2.4.19 Linux introduced the concept of namespaces to virtualize such resources. This paper will give a brief 
introduction to Linux namespaces and present KIT a tool that was developed by Congyu Liu et al. to
find bugs in those.
\end{abstract}

\section{Introduction}\label{sec:introduction}
Linux namespaces are a kernel feature that allows process isolation on the operating system level.
This allows to virtualize resources like the network stack or the process tree.
While processes in the same namespace see the modifications their peers are making to system resources
processes in other namespaces do not. Unlike similar technologies namespaces were designed with
isolation in mind. Thus it's no suprise that namespaces are widely used and build the foundation
for many container technologies like podman, docker or lima.
KIT is a new framework developed by Congyu Liu et al. for finding bugs in the implementation of
security checks that are needed for linux namespaces\cite{0}. 
This is done by comparing system call traces of processes with and without possible interference.
This report gives a brief introduction to the inner
workings of KIT and a detailed description of one bug
that was found by KIT. 

\section{Linux Namespaces}
\subsection{Usecases}
Containers create a reproducible environment that is independent of the host system with minimal overhead.
Thus, software that is run inside a container is not affected if the host system changes, 
container engines like docker or podman achieve this by using namespaces and other isolation
technologies to build such environments.


Another usecase for namespace is sandboxing. For example Chrome and Firefox use Linux namespaces
in their site isolation sandboxes. 
Chrome uses PID namespaces and user namespaces to isolate certain processes.
This is done to reduce the impact of vulnerabilities in those processes that could lead to remote code execution.


\subsection{Inner workings}
Linux knows 7 namespaces: mount, PID, IPC, UTS, net, user and cgroup namespaces. Each namespace isolates its own set of resources. Namespaces are implemented in the Linux kernel.
Each process has a link to the namespaces it belongs to. This link can be found in /proc/self/ns\cite{8}. The namespaces are implemented as a tree. Each namespace has a parent and 
elements in child namespaces are visible in the parent namespace, there they can show different
attributes than in their respective child namespaces. For example a process in a PID namespace might 
believe that it has pid 1 while the PID in the parent namespace has 21 and the PID in the root
namespace has 42.

Linux handles namespaces via inodes. Each namespace has a corresponding inode. Processes in this
namespace link to this inode. If no links to the inode exist anymore and thus no processes are in
this namespace the namespace is destroyed. To keep a namespace alive even with no process in it
one can bind the corresponding file to another place in the filesystem. To reference a namespace a
filedescriptor is used. Namespaces are managed by the kernel processes can use various system
calls to create new namespaces or to join existing ones.\cite{8} Following is a list of the seven
kinds of namespaces and their features.


\subsection{Process ID namespace}
The PID namespace isolates the set of process id's. 
Thus each PID namespace has a full set of PID's to use and  the PIDs can start with one.
Processes in different PID namespaces can have the same PID in their respective namespace. 
Processes that are spawned in a PID namespace get a PID in the parent PID namespace as well 
\cite{11}.

\subsection{Network namespaces}
Each network namespace has it's own network stack. Therefor each net namespace has its own set of ip
addresses, firewall rules, routing tables and sockets. Physical network hardware is assigned to exactly one
net namespace and can only be used from within this namespace. Linux provides a virtual network
devices that can connect net namespaces.\cite{10}

\subsection{Mount namespaces}
Mount namespaces isolate the filesystem mount points. 
Each mount namespace provides a unique and seperate view of the system mount points. 
Upon creation these views are copied from the parent or calling processes namespace.
To soften the impacts of isolation Linux introduced shared subtrees. This allows the propagation of
mount and unmount events between namespaces.\cite{12}

\subsection{IPC namespaces}
IPC namespaces isolates POSIX message queues and System V IPC objects. This allows for the reuse of identifiers
in different namespaces.\cite{13}

\subsection{UTS namespaces}
UTS namespaces isolate the hostname and domain name. Upon creation the hostname and domain name are copied from the parent namespace.\cite{14}

\subsection{Time namespaces}
Time namespaces isolate the system clock. This allows different for offsets for different namespaces.
Unlike other namesapces this namespace can only be created via unshare and the calling process is
not placed inside the namespace.\cite{15}

\subsection{Cgroup namespaces}
Cgroups are a kernel feature that allows  grouping processes and monitoring or limiting the resource usage
of such groups. Resouces usually regulated are memory and cpu time \cite{16}. Cgroups are
organized in a tree hierachy. The cgroup namespace isolates the view of the whole system to
that subtree visible to the cgroup itself. This means that a process
in the cgroup namespaces sees his own cgroup as the root of the cgroup tree. \cite{17}

\subsection{User namespaces}
User namespaces isolate user ids, group ids and permission related attributes like keyrings and
capabilities. User namespaces are nested so that all such user namespaces except the root user namespace
have a parent namespace. The parent namespace is the namespace of the creating process. 
It is possible to map user and group ids in a user namespace, this means that a process in a new user
namespace can have user id 0 and group id 0 in this namespace while not being root. The same mapping
can be done for euid and egid's as well \cite{18}. The second method the kernel uses to keep track
of permissions are capabilities. Capabilities are a way to split the privileges of root into smaller
chunks. With capabilities it is possible to give a process a subset of the root
privileges. For example a process that holds the CAP\_NET\_ADMIN capability can access network
operations while a process holding CAP\_KILL is permitted to send signals to other
processes\cite{19}. CAP\_SYS\_ADMIN is a capability which covers some of the
other capabilities. 
User namespaces do also isolate these capabilities. For example a process that creates a new
namespace and joins this namespace holds the CAP\_SYS\_ADMIN capability in the namespace even if it did not before. Obviously the CAP\_SYS\_ADMIN does not grant the same privileges one would expect from
a  CAP\_SYS\_ADMIN but it is still enough to subvert the idea of capabilities, confuse users and
developers and even increase the attack surface of the kernel criticized \cite{20}. 

\subsection{Namespace usage}
As previously mentioned processes interact with namespaces via system calls. There exist 3 system calls
to manipulate namespaces. 
\subsection{ clone}
The clone system call creates a new child process. The child process can be placed in a new
namespace by passing the corresponding flags. This however requires the calling process to hold
CAP\_SYS\_ADMIN. For example to create a new PID namespace the CLONE\_NEWPID flag is used like this:
\begin{lstlisting}[language=C]
    int pid = clone(child_func, child_stack + STACK_SIZE, CLONE_NEWPID | SIGCHLD, NULL);
\end{lstlisting}
The pid the parent process gets is in the parent namespace and therefor different from the pid the
child process would see if it would call getpid(). (Note that clone returns pid 0 in the
child)\cite{22}.

\subsection{ setns}
If a process wants to join an existing namespace it can use the setns system call. This system call
allows a process to join a namespace by passing a file descriptor to a link of the namespace. The
process can also specify what type if namespace it wants to join by providing a flag. The
requirments for this system call depend on the provided arguments. For user namespaces the process
must hold CAP\_SYS\_ADMIN in the target namespace since it is granted CAP\_SYS\_ADMIN upon joining
regardless of its group's or users. This realistically means that a process can only
join descendant namespaces. Furthermore a process can not rejoin his own namespace since this would
allow to recover dropped capabilities \cite{23}. The following code snippet shows how to join the
cgroup namespace of process 42:
\begin{lstlisting}[language=C]
    int fd = open("/proc/42/ns/cgroup", O_RDONLY);
    setns(fd, CLONE_NEWCGROUP);
    //setns(fd, 0); //would join any namespace
\end{lstlisting}


\subsection{ unshare}
The unshare system call allows a process to create a new namespace and place itself in this
namespace (except for  time namespaces where childes would be placed in the namespace \cite{15}). 
The process can specify what type of namespace it wants to create by passing a flag. To create a new
UTS namespace the following code can be used:
\begin{lstlisting}[language=C]
    unshare(CLONE_NEWUTS);
\end{lstlisting}\cite{24}
\section{Related Technologies}\label{sec:related}
\subsection{chroot}
Linux knows the change root system call which changes the root directory of the calling
process. This means that the calling process and its children can not access files outside of the
new root. This is implemented in the pathname resolution\cite{1}. Chroots were not intended as
isolation technique and thus can be escaped sometimes. Linux mount namespaces are
designed as isolation tool and can achieve the same results as chroot. Besides that one can use
chroots in combination with mount namespaces\cite{1},\cite{25}. 

\subsection{SELinux and AppArmor}
SELinux and AppArmor are systems that allow the implementation of mandatory access control. In
contrast to the traditional Linux access control this means that policies are enforced by the system
and can not be altered by the user. Furthermore these tools allow a more fine grained control over
access rights. This can be done to protect other processes and the system but does not allow the resource
virtualization that namespaces allow\cite{26}\cite{27}. 
\subsection{Windows containers}
Windows seems to have a similar concept to Linux namespaces called process isolation. Not much is known about the
implementation. Additionally windows allows also so called hyper-v isolation, here each container
gets his own virtual machine and is run on the hypervisor\cite{3}.
\section{Namespace bug}
\subsection{Interference bugs}
Many bugs in the Linux namespace implementation are semantic bugs. This means that the 
bug does not result in invalid memory access or a crash but in wrong behavior. Functional interference bugs are
such bugs that allow processes to interfere with processes outside of their namespace. Thus
breaking isolation. This bugs are hard to detect with traditional testing methods due to the
complexity of the bugs and the complexity of the kernel itself.
\section{KIT}
KIT is a new framework for finding interference bugs in OS level vitalization \cite{0}. It was
implements a new differential approach to finding such bugs and found 9 new bugs in kernel 5.13\cite{2}.
KIT is designed as pipeline and implemented in 7600 lines of code not counting existing tools\cite{0}.
The main idea behind KIT is to check if a process behaves differently if another process is running
in parallel.
\subsection{Testcase generation}
KIT takes as input a corpus of testcases these come from other kernel fuzzers.
A KIT testcase contains to programs, one which sender who modifies a namespace protected kernel
resource and a receiver who reads from the same resource. To find such testcases
KIT runs the input testcases in a special environment and profiles their memory access. 
After that 2 programs are paired together if they access similar memory regions with a write and
read access and the memory region belongs to an namespace protected resource.
To reduce the number of testcases uses heuristics to group similar testcases together.


\subsection{Testcase execution}
KIT uses Qemu to run testcases in virtual machines. To avoid non deterministic behavior KIT uses
a different communication channel and reloads vm snapshots after each testcase. KIT executes a
testcase twice. The first time first the sender program is run and then the receiver program. The
system call trace of the receiver is collected. In the second run the sender is skipped and the receiver
is run again to collect a baseline system call trace\cite{0}. 
\subsection{Bug detection}
KIT keeps these system call traces as Abstract Syntax Trees (AST). This additional information allows 
to create minimal testcases and to filter out irrelevant system calls. This is done by comparing the ASTs
recursively and ignoring children that are flagged as irrelevant. These flags are typically set when
a system call involves resources that are not isolated by namespaces or if the output of the
system call is not static. This can be found out by rerunning the receiver multiple times and
comparing the system call traces. Each inner node that differs in these AST's can be flagged as non
deterministic.
After a difference is found KIT creates a minimal by differentially removing system calls. This
minimal testcase is then passed to the next stage\cite{0}.
\subsection{report aggregation}
The last stage of KIT is the report aggregation. This stage is important since the results will be
analyzed by humans and thus need to be reduced to a minimum to save time. KIT does this by grouping
testcases together if they have the same sender and receiver system calls that are responsible for
the bug\cite{0}. 

\section{Related Work}
Currently there is no comparable tool like KIT. Other tools like syzkaller\cite{4} and trinity\cite{5} are
unlikely to find such bugs since they lack the ability to detect anomalies in the behavior of
namespace protected resources and often do not reach deep enough in the code base.
The static approach for finding interference bugs seems not feasible
since due to the complexity of the kernel and the nature of the bugs many false positives would
be found. KIT makes use of some existing tools like syzkaller and needs existing testcases to
perform its analysis\cite{0}\cite{2}.
\section{A namespace bug}\label{sec:bug}
Personally I was not able to reproduce the bug. But
since i could not get kit to run and thus used syzkallers POC extractor I assume that the mistake
happened in the conversion from syzkaller to C code\cite{0}.
\subsection{Bug description}
The paper found 9 new bugs in the Linux kernel 5.13 \cite{0}. One of these bugs is a net namespace
bug that will be described in this section. 
\subsection{Network background}
Sockets are pseudo files that serve as endpoints for communication similar to pipes. Unlike pipes
they allow duplex communication and can allow communication over networks\cite{5}. Sockets are
used to implement network protocols in user space \cite{5}. The kernel forwards the packets to the
corresponding socket. This mapping can be seen in /proc/net/ptype \cite{6}. Linux sockets are
namespace protected and thus no information about them should be shared between namespaces.

\subsection{Bug description}
Packets that are bound to devices are only showed in the namespace of the device. However if packet
types are not bound to a device they have no reference to a namespace and are shown in all network
namespaces. This allows a process to create a packet type in a network namespace and a second
process can see this packet type in a different network namespace thus breaking isolation.


\subsection{Bug reproduction}
Analysis is done on the customized kit kernel 5.6 which is was from the artifact repository\cite{2}.
Sadly the kernel that kit used is not easily setup for custom debugging.
The test case that caused the bug given from syzcaller is the following: 
\begin{lstlisting}{caption=sender test case}
r0 = socket$packet(0x11, 0x2, 0x300)
bind$packet(r0, 
    &(0x7f0000000100)={0x11, 0xc, 0x0, 0x1, 0x0, 0x6,
    @multicast}, 0x14)
bind$packet(r0, 
&(0x7f0000000040)={0x11, 0x7, 0x0, 0x1, 0x0, 0x6,
@broadcast}, 0x14)
\end{lstlisting}
The receiver test case can also be taken from syzcaller but the process is tedious error prone and
we already now that the leakage happens in the /proc/net/ptype. Thus we can use cat.
Syzcaller provides tools to convert the testcase to C code. 
\begin{lstlisting}
// autogenerated by syzkaller (https://github.com/google/syzkaller)

#define _GNU_SOURCE

#include <endian.h>
#include <stdint.h>
#include <stdio.h>
#include <stdlib.h>
#include <string.h>
#include <sys/syscall.h>
#include <sys/types.h>
#include <unistd.h>

uint64_t r[1] = {0xffffffffffffffff};

int main(void)
{
  syscall(__NR_mmap, 0x1ffff000ul, 0x1000ul, 0ul, 0x32ul, -1, 0ul);
  syscall(__NR_mmap, 0x20000000ul, 0x1000000ul, 7ul, 0x32ul, -1, 0ul);
  syscall(__NR_mmap, 0x21000000ul, 0x1000ul, 0ul, 0x32ul, -1, 0ul);
  intptr_t res = 0;
  res = syscall(__NR_socket, 0x11ul, 2ul, 0x300);
  if (res != -1)
    r[0] = res;
  *(uint16_t*)0x20000100 = 0x11;
  *(uint16_t*)0x20000102 = htobe16(0xc);
  *(uint32_t*)0x20000104 = 0;
  *(uint16_t*)0x20000108 = 1;
  *(uint8_t*)0x2000010a = 0;
  *(uint8_t*)0x2000010b = 6;
  memset((void*)0x2000010c, 187, 6);
  memset((void*)0x20000112, 0, 2);
  syscall(__NR_bind, r[0], 0x20000100ul, 0x14ul);
  *(uint16_t*)0x20000040 = 0x11;
  *(uint16_t*)0x20000042 = htobe16(7);
  *(uint32_t*)0x20000044 = 0;
  *(uint16_t*)0x20000048 = 1;
  *(uint8_t*)0x2000004a = 0;
  *(uint8_t*)0x2000004b = 6;
  memset((void*)0x2000004c, 255, 6);
  memset((void*)0x20000052, 0, 2);
  syscall(__NR_bind, r[0], 0x20000040ul, 0x14ul);
  return 0;
}
\end{lstlisting}
Obviously this code does not work after the process terminates its artifacts are removed and thus we
add an sleep at the bottom of the main method. The code can be compiled and run (with unshare -n and root if this
is necessary is not in the scope of this analysis). After the code is run one can see the following
in /proc/net/ptype:
\begin{lstlisting}
Type Device      Function
0800          ip_rcv
0806          arp_rcv
0007          packet_rcv // this is created by our poc.
86dd          ipv6_rcv
\end{lstlisting}
Thus the POC works. Sadly syzcaller does not comply with any coding standards and thus the code is
not very readable. However using strace, r2\cite{29} and gdb\cite{28} we can "reverse" the code quite easily and get this
cleaned up version:
\begin{lstlisting}{caption=cleaned up poc}
/*...*/
int main() {
    struct sockaddr  a = {.sa_family = AF_PACKET,.sa_data= "\000\f\000\000\000\000\001\000\000\000\000\000\000"};
    struct sockaddr  b = {.sa_family = AF_PACKET,.sa_data= "\000\a\000\000\000\000\001\000\000\006\377\377\377\377"};
    int sock = socket(AF_PACKET, SOCK_DGRAM, htons(ETH_P_ALL));

    bind(sock, &a, 20);
    bind(sock , &b, 20);
    return 0;
}
\end{lstlisting}
While this might not be the exact code that syzcaller gave us, we can run it and check if it still
produces the same results. And indeed it does. Now we can try to reduce the test case. And indeed
only one bind is necessary to trigger the bug. The data in the sockaddr struct seems to be arbitrary
as long as it does not describe a valid device. 
\subsection{Analysis of the code}
Thanks to the patch we know the responsible method.
The responsible method can be found in net/core/net-procfs.c. It is ptype\_seq\_show
and here is the important code snippet:
\begin{lstlisting}{caption=vulnerable code snippet}
/*...*/
else if (pt->dev == NULL || dev_net(pt->dev) == seq_file_net(seq)) {
    /* show packet types */
\end{lstlisting}
Normally packet types are bound to a device. If this method is invoked it is checked if the device
associated with the packet type is in the same network namespace as the calling process. This is
done by comparing the network namespace of the device with the network namespace of sequence file.
However to avoid null pointer exceptions short circuiting is used to check if the device is null. If
this is the case the namespace check is skipped. The following syzcaller test case does this.
The leakage can be seen by reading the file /proc/net/ptype in a different network namespace for
example with cat.

\subsection{Bug fix}
The fix for this bug is adding a pointer to the network namespace that references the namespace
where the packet type was created. This pointer is then checked whenever the packet type is
viewed. A fix has been submitted.\cite{4}
The previous code snippet is changed to the following:
\begin{lstlisting}{caption=fix}
/*...*/
	else if ((!pt->af_packet_net ||  net_eq(pt->af_packet_net, seq_file_net(seq))) &&
		 (!pt->dev || net_eq(dev_net(pt->dev), seq_file_net(seq)))) {
                 /* show packet types */
\end{lstlisting}
\cite{4}
\subsection{Impact}
This bug presents a minor information leakage and is merely relevant for desktop systems.
However in an shared host environment this bug allows two processes to check if they run on the same
hardware. This information can be used to perform power attacks. These attacks aim to create power
outages by creating power peaks in multiple hosts at the same time. If two containers share the same
host this knowledge can decrease the cost of such attacks\cite{7}.
\section{Conclusion}
Namespaces are a powerful established way to isolate processes. They are used in many different
environments 
and are a key component of containerization and sandboxing. Thus research in the field of namespace isolation is
important. KIT implements a new approach to namespace fuzzing and was able to find new bugs in the
kernel, which proves the effectiveness of this approach. It remains to be seen what differential
fuzzing can find with a custom  test corpus and further optimizations. It is currently unclear if
kit will be adopted by the community or not. That being said and the  grown interest in formal
verification in recent years in mind one may look forward to a formal verification of the Linux
kernel which seems to be the natural way of finding such logic bugs in the first place.
\bibliographystyle{abbrv}
\bibliography{template}
%\footnotesize
\end{document}
