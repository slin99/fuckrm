%%
%%  Example paper
%%
%%

%%%%%%%%%%%%%%%%%% Usenix style %%%%%%%%%%%%%%%%%%%%%%%%%%%%%%%%%
\documentclass[10pt,twocolumn,a4paper]{article}
\usepackage{styles/usenix-style}

\author{Jean Diestl}

%%%%%%%%%%%%%%%%%% Document %%%%%%%%%%%%%%%%%%%%%%%%%%%%%%%%%%%%%%%%%%%
% TODO: Change draft to final before submitting final version.
\usepackage[draft]{styles/ka-style}
\usepackage{cite,xspace,ifthen,graphicx,listings}

\usepackage[
   pdfauthor={Jean Diestl},
   pdftitle={Linux namespaces},
   pdfsubject={KIT testing OS Level Virtualization},
   pdfkeywords={Papers, Template}
]{hyperref}

\begin{document}

\title{Linux Namespaces: KIT testing OS Level Virtualization}

\newcommand{\todo}[1]{{\texttt{[#1]}}}
\newcommand{\code}[1]{{\tt \small{#1}}}

\maketitle
%\draftfooter

\begin{abstract}
Bla.
\end{abstract}

\section{Introduction}\label{sec:introduction}
\section{Related Technologies}\label{sec:related}
\subsection{chroot}
Linux knows the change root system call which changes the root directory of the calling
process. This means that the calling process and its children can not access files outside of the
new root. This is implemented in the pathname resolution\cite{1}. Chroots were not intended as
isolation technique and thus can be escaped sometimes. Linux mount namespaces are
designed as isolation tool and can achieve the same results as chroot. Besides that one can use
chroots in combination with mount namespaces. %TODO cite source for escape 


\subsection{Windows containers}%TODO change name
Windows seems to have a similar concept to linux namespaces called process isolation. Not much is known about the
implementation. Additionally windows allows also so called hyper-v isolation, here each container
gets his own virtual machine and is run on the hypervisor\cite{3}.

\section{Namespace bug}\label{sec:background}
\subsection{Interference bugs}
Many bugs in the linux namespace implementation are semantic bugs. This means that the 
bug does not result in invaldi memory access or a crash but in wrong behavior. Functional interference bugs are
such bugs that allow processes to interfere with processes outside  of their namespace. Thus
breaking isolation. This bugs are hard to detect with traditional testing methods due to the
complexity of the bugs and the complexity of the kernel itself.
\section{Related Work}\label{sec:relwork} 
\section{KIT}
KIT is a new framework for finding interference bugs in OS level virtualization\cite{0}. It was
implements a new differential approach to finding such bugs and found 9 new bugs in kernel 5.13\cite{2}.
KIT is designed as pipeline and implemented in 7600 lines of code not counting existing tools\cite{0}.
The main idea behind KIT is to check if a process behaves differently if another process is running
in parallel.
\subsection{Testcase generation}
KIT takes as input a corpus of testcases these come from other kernel fuzzers.
A KIT testcase contains to programs, one which sender who modifies a namespace protected kernel
resouce and a reciever who reads from the same resource. To find such testcases
KIT runs the input testcases in a special environment and profiles their memory access. 
After that 2 programms are paired together if they access similar memory regions with a write and
read access and the memory region belongs to an namespace protected resouce.
To reduce the number of testcases uses heuristics to group similar testcases together.


\subsection{Testcase execution}
KIT first runs the reciever program multiple times to get a baseline for the result. After that
the sender program is run in parallel to the reciever program. Depending on instrumentation used
this process can be repeated multiple times since instrumentation can change the behavior of the
programs. The output of the reciever that is passed to the next stage is the systemcall trace.
\subsection{Bug detection}
KIT keeps these systemcall traces as Abstract Syntax Trees (AST). This additional information allows 
to create minimal testcases and to filter out irrelevant systemcalls. This is done by comparing the ASTs
recursivly and ignoring children that are flagged as irrelevant. These flags are typically set when
a systemcall involves resources that are not isolated by namespaces or if the output of the
systemcall is not static. This can be foundout during the execution when a baseline is created.
After a difference is found KIT creates a minimal by differntially removing systemcalls. This
minimal testcase is then passed to the next stage.
\subsection{report aggregation}
The last stage of KIT is the report aggregation. This stage is important since the results will be
analyzed by humans and thus need to be reduced to a minimum to save time. KIT does this by grouping
testcases together if they have the same sender and reciever systemcalls that are responsible for
the bug. 
\section{A namespace bug}\label{sec:bug}
\subsection{Bug description}
The paper found 9 new bugs in the linux kernel 5.13 \cite{0}. One of these bugs is a net namespace
bug that will be described in this section. 

\subsection{Network background}
Sockets are pseudo files that serve as endpoints for communication similar to pipes. Unlike pipes
they allow duplex communication and can allow communication over networks\cite{5}. Sockets are
used to implement network protocols in user space \cite{5}. The kernel forwards the packets to the
corresponding socket. This mapping can be seen in /proc/net/ptype \cite{6}. Linux sockets are
namespace protected and thus no information about them should be shared between namespaces.

\subsection{Bug description}
Packets that are bound to devices are only showed in the namespace of the device. However if packet
types are not bound to a device they have no reference to a namespace and are shown in all network
namespaces. This allows a process to create a packet type in a network namespace and a second
process can see this packet type in a different network namespace thus breaking isolation.
%Maybe code 
\subsection{Bug fix}
The fix for this bug is adding a pointer to the network namespace that references the namespace
where the packet type was created. This pointer is then check whenever the packet type is
viewed. A fix has been submitted.\cite{4}
\subsection{Impact}
This bug presents a minor information leakage and is merely relevant for desktop systems.
However in an shared host enviroment this bug allows two processes to check if they run on the same
hardware. This information can be used to perform power attacks. These attacks aim to create power
outages by creating power peaks in multiple hosts at the same time. If two containers share the same
host this knowledge can decrease the cost of such attacks\cite{7}.
\section{Conclusion}\label{sec:conclusion}

\bibliographystyle{abbrv}
\bibliography{template}
%\footnotesize
\end{document}
